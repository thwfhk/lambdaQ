\section{Usage and Examples}

See the \texttt{readme} file of our repository for code structure and usage.
Here we give some examples of $\lambda_Q$ program.
Note that a $\lambda_Q$ program should contain at least one circuit abstraction (i.e. $\kappa$ abstraction) as its entry point, just like the main function in other programming languages.
For simplicity, we assume the last circuit abstraction is the entry point.


\subsection{Quantum Teleportation}

This is a classical example also appeared in some other papers about quantum programming languages.
It can be written in $\lambda_Q$ like follows.

\begin{lstlisting}[language=Lambda]
fun bell00 = / () : One .
  a <- gate init0 ();
  b <- gate init0 ();
  a <- gate H a;
  (a, b) <- gate CNOT (a, b);
  output (a, b)

fun alice = / (q, a) : Qubit # Qubit .
  (q, a) <- gate CNOT (q, a);
  q <- gate H q;
  x <- gate meas q;
  y <- gate meas a;
  output (x,y)

fun bob = / ((w1, w2), q) : Bit # Bit # Qubit .
    (x1, x2) <-| lift (w1, w2);
    q <- capp (if x2
               then (/ t : Qubit . gate X t)
               else (/ t : Qubit . output t)
         ) to q;
    capp (if x1
          then (/ t : Qubit . gate Z t)
          else (/ t : Qubit . output t)
    ) to q

fun teleport = / () : One .
  q <- gate init0 ();
  (a, b) <- capp bell00 to ();
  (x, y) <- capp alice to (q, a);
  capp bob to ((x, y), b);
\end{lstlisting}

The output of the frontend is:
\begin{lstlisting}[language=lambda]
openqasm 2.0;
qreg r0[1];
qreg r1[1];
qreg r2[1];
H r1;
CX r1, r2;
CX r0, r1;
H r0;
creg r3[1];
measure r0 -> r3;
creg r4[1];
measure r1 -> r4;
if (r4 == 1) X r2;
if (r3 == 1) Z r2;
\end{lstlisting}

And the output of the backend is:
\begin{lstlisting}[language=lambda]
openqasm 2.0;
qreg q[2];
CX q[1], q[0];
\end{lstlisting}

The optimization really works!

\subsection{Simple Optimization}
Here is another simple example to show the optimization power of our backend.
The original $\lambda_Q$ program is:
\begin{lstlisting}[language=lambda]
fun qwq = / () : One .
  a <- gate init0 ();
  b <- gate init0 ();
  a <- gate H a;
  b <- gate H b;
  (c, d) <- gate CNOT (a, b);
  c <- gate H c;
  d <- gate H d;
  output (c, d)
\end{lstlisting}

The output of the frontend is:
\begin{lstlisting}[language=lambda]
openqasm 2.0;
qreg r0[1];
qreg r1[1];
H r0;
H r1;
CX r0, r1;
H r0;
H r1;
\end{lstlisting}

And the output of the backend is:
\begin{lstlisting}[language=lambda]
openqasm 2.0;
qreg q[2];
CX q[1], q[0];
\end{lstlisting}

The optimization really works!