\section{Specification of $\lambda_Q$}

In this section we give the language specification of $\lambda_Q$, including the syntax and typing rules.
The syntax and typing rules of the quantum circuit are similar to Qwire \cite{qwire}.
The operational semantics are not given in this section.
Instead, we give a transformation from $\lambda_Q$ to $QASM$ in the section of frontend, which gives semantics to each syntax of $\lambda_Q$.
Interested readers can refer to the operational semantics of simply-typed $\lambda$-calculus and Qwire.

\subsection{Syntax}
The syntax of the traditional part (terms, values, types) is an extended version of simply-typed $\lambda$-calcus with $\trun\ C$ and $\kappa \ p:W . C$, which are used to interact with the quantum part.
The syntax of the quantum part (circuits, wire types, wire patterns, gates, gate types) is similar to the syntax of Qwire.
% \mycomment{twh}{modify syntax (especially gate)}

\begin{longtable}[c]{lclr}
  % \caption{syntax}
  \label{tab:table1}\\
  \toprule
  $t$ &$::=$ &  &\textbf{terms}: \\
      & &$x$ &variable\\
      & &$\unit$ &constant unit\\
      & &$\true$ &constant true\\
      & &$\false$ &constant false\\
      & &$\lambda\ x:T.t$ &function abstraction\\
      & &$t\ t$ &function application\\
      & &$(t, t)$ &pair\\
      & &$t.1$ &first projection\\
      & &$t.2$ &second projection\\
      & &$\tif\ t\ \tthen\ t\ \telse\ t$ &conditional\\
      & &$\trun\ C$ &static lifting\\
      & &$\kappa \ p:W . C$ &circuit abstraction\\
      % & &0 &constant zero\\
      % & &succ $t$ &successor\\
      % & &pred $t$ &predecessor\\
      % & &iszero $t$ & zero test\\
  \\
  
  $v$ &$::=$ &  &\textbf{values}: \\
      & &$\lambda\ x:T.t$ &abstraction value\\
      & &$(v, v)$ &pair value\\
      & &$\unit$ &unit value\\
      & &$\true$ &true value\\
      & &$\false$ &false value\\
      & &$\kappa\ p:W . C$ &circuit value\\
  \\

  $T$ &$::=$ &  &\textbf{types}: \\
      & &$\Unit$ &unit type\\
      & &$\Bool$ &boolean type \\
      & &$T\times T$ &product type\\
      & &$T\to T$ &function type\\
      & &$T\leadsto T$ &circuit type\\
      % & &Nat &type of natural numbers\\
      % & &Vector $t$ &type family of vectors\\
  \\

  $\Gamma$ &$::=$ &  &\textbf{contexts}: \\
      & &$\varnothing$ &empty context\\
      & &$\Gamma,x:T$ &term variable binding\\
  \\


  $W$ &$::=$ &  &\textbf{wire types}: \\
      & &$\One$ &wire unit type\\
      & &$\Bit$ &bit type \\
      & &$\Qubit$ &qubit type \\
      & &$W \otimes W$ &wire product type \\
  \\

  $p$ &$::=$ &  &\textbf{wire patterns}: \\
      & &$()$ &empty\\
      & &$w$ &wire variable \\
      & &$(p,p)$ &wire pair \\
  \\

  $C$ &$::=$ &  &\textbf{circuits}: \\
      & &$\toutput\ p$ &output a pattern \\
      & &$p_2 \from \gate\ g\ p_1 ; C$ &gate application \\
      & &$p \from C ; C$ &circuit composition \\
      & &$x \hookleftarrow \lift\ p ; C$ &dynamic lifting \\
      & &$\capp\ t\ \mathtt{to}\  p$ &circuit application \\
  \\

  $g$ &$::=$ &  &\textbf{gates}: \\
      & &$\new_0$ &generate a bit 0 \\
      & &$\new_1$ &generate a bit 1 \\
      & &$\init_0$ &generate a qubit 0 \\
      & &$\init_1$ &generate a qubit 1 \\
      & &$\meas$ &measurement gate \\
      & &$\discard$ &disgard gate \\
      & &$\Ha$ &Hadamard gate \\
      & &$\X$ &Pauli-X gate \\
      & &$\Z$ &Pauli-Z gate \\
      & &$\CNOT$ &CNOT gate \\
  \\

  $G$ &$::=$ &  &\textbf{gate types}:\\
      & &$\mathcal{G}(W, W)$ &simple gate type\\
  \\

  $\Omega$ &$::=$ &  &\textbf{wire contexts}: \\
      & &$\varnothing$ &empty context\\
      & &$\Omega,w:W$ &wire variable binding\\
  \\
  % TODO: add the condition of well-formed contexts

  \bottomrule
  
\end{longtable}

\subsection{Typing Rules}
The main feature of the language design of $\lambda_Q$ is to use linear type system to guarantee no quantum bit is used twice or not used at all.
To state the type inference rules of $\lambda_Q$, we first need to define what is a \textbf{well-formed wire context}, which is used to maintain variables used by quantum circuit in the calculus.
This context is actually corresponding to the context of linear variables in the traditional linear type system.
\begin{Def}[Well-formed Wire Contexts]
  A wire context $\Omega$ is well-formed, if there are no duplicate wire variables in it.
  For simplicity, we always assume the wire contexts are well-formed in the following contexts. And when we write $\Omega_1, \Omega_2$, we require $\Omega_1$ and $\Omega_2$ to be disjoint to preserve the well-formedness.
\end{Def}

Since there are some different kinds of terms: ($\lambda$-)terms, wire patterns, gates and circuits, we have defined several different typing relations for each of them.
\begin{itemize}
  \item $\Omega \vdash p : W$ is the typing relation for patterns.
  \item $\Gamma ; \Omega \vdash C : W$ is the typing relation for circuits;
  \item $\Gamma \vdash t:T$ is the typing relation for ($\lambda$-)terms;
  \item $g : G$ is the typing relation for gates.
\end{itemize}

\subsubsection{Typing rules for gates}
Note that since we only support built-in gates, the typing rules for gates are extremely simple, just assigning a type to each built-in gate.

\noindent \textbf{Typing rules for gates}: $\boxed{g : G}$
\renewcommand\arraystretch{2.5}
\begin{longtable}[c]{cr}
  $ \infer{\new_0 : \calG(\One, \Bit)}{}$ & \\
  $ \infer{\new_1 : \calG(\One, \Bit)}{}$ & \\
  $ \infer{\init_0 : \calG(\One, \Qubit)}{}$ & \\
  $ \infer{\init_1 : \calG(\One, \Qubit)}{}$ & \\
  $ \infer{\meas : \calG(\Qubit, \Bit)}{}$ & \\
  $ \infer{\discard : \calG(\Bit, \One)}{}$ & \\
\end{longtable}

\subsubsection{Typing rules for patterns}
The patterns are used to construct complex gates.
Patterns are destructed when doing pattern matching in gate application, circuit composition and dynamic lifting.

\noindent \textbf{Typing rules for patterns}: $\boxed{\Omega \vdash p : W}$
\renewcommand\arraystretch{2.5}
\begin{longtable}[c]{cr}
  $ \infer{\varnothing \vdash ():One}{}$ & \\
  $ \infer{w:W \vdash w:W}{}$ & \\
  $ \infer{\Omega_1, \Omega_2 \vdash (p_1,p_2):W_1\otimes W_2}{\Omega_1 \vdash p_1 : W_1  &\Omega_2 \vdash p_2 : W_2} $ & \\
\end{longtable}

\subsubsection{Typing rules for circuits}
The typing rules for circuit uses linear type. The context $\Gamma$ is for normal (traditional) variables and the context $\Omega$ is for linear (quantum) variables.

\noindent \textbf{Typing rules for circuits}: $\boxed{\Gamma;\Omega\vdash C:W}$


\renewcommand\arraystretch{3} 
\begin{longtable}[c]{cr}
  $\infer{\Gamma;\Omega\vdash \toutput\ p:W}{\Omega\vdash p : W}$ &(C-OUTPUT)\\
  $\infer{\Gamma;\Omega_1,\Omega\vdash p_2 \from \gate\ g\ p_1 ; C:W}{g:\calG(W_1, W_2) &\Omega_1\vdash p_1:W_1 &\Omega_2\vdash p_2:W_2 &\Gamma ; \Omega_2,\Omega \vdash C:W}$ &(C-GATE)\\
  $\infer{\Gamma;\Omega_1,\Omega_2\vdash p \from C ; C':W'}{\Gamma;\Omega_1\vdash C:W &\Omega \vdash p:W &\Gamma;\Omega,\Omega_2 \vdash C':W'}$ &(C-COMPOSE)\\
  $\infer{\Gamma;\Omega,\Omega'\vdash x \hookleftarrow \lift\ p ; C:W'}{\Omega\vdash p:W &\Gamma,x:|W|;\Omega' \vdash C:W'}$ &(C-LIFT)\\
  $\infer{\Gamma;\Omega\vdash \capp\ t\ \mathtt{to}\ p : W_2}{\Gamma\vdash t:W_1\leadsto W_2 &\Omega\vdash p:W_1}$ &(C-CAPP)\\
  
\end{longtable}

\subsubsection{Typing rules for terms}

\noindent \textbf{Typing rules for ($\lambda$-)terms}: $\boxed{\Gamma\vdash t:T}$

\renewcommand\arraystretch{3}
\begin{longtable}[c]{cr}
  $\infer{\Gamma\vdash \trun\ C : |W| }{\Gamma;\varnothing \vdash C:W}$ &(T-RUN)\\
  $\infer{\Gamma\vdash \kappa\ p:W_1 . C: W_1\leadsto W_2 }{\Omega\vdash p:W_1 &\Gamma;\Omega\vdash C:W_2}$ &(T-CABS)\\
  $\infer{\Gamma\vdash x:T}{x:T\in \Gamma}$ &(T-VAR)\\
  $\infer{\Gamma\vdash \lambda\ x:T_1.t_2 : T_1\to T_2}{\Gamma,x:T_1 \vdash t_2:T_2}$ &(T-ABS)\\
  $\infer{\Gamma\vdash t_1\ t_2 : T_{12}}{\Gamma\vdash t_1:T_{11}\to T_{12} & \Gamma\vdash t_2:T_{11}}$ &(T-APP)\\
  $\infer{\Gamma\vdash\true : Bool}{}$ &(T-TRUE)\\
  $\infer{\Gamma\vdash\false : Bool}{}$ &(T-FALSE)\\
  $\infer{\tif\ t_1\ \tthen\ t_2\ \telse\ t_3 : T}{\Gamma\vdash t_1:\Bool &\Gamma\vdash t_2:T &\Gamma\vdash t_3:T}$ &(T-IF)\\
  $\infer{\Gamma\vdash\unit : Unit}{}$ &(T-UNIT)\\
  $\infer{\Gamma\vdash (t_1,t_2) : T_1\times T_2}{\Gamma\vdash t_1:T_1 &\Gamma\vdash t_2:T_2}$ &(T-PAIR)\\
  $\infer{\Gamma\vdash t.1 : T_1}{\Gamma\vdash t : T_1\times T_2}$ &(T-FST)\\
  $\infer{\Gamma\vdash t.2 : T_2}{\Gamma\vdash t : T_1\times T_2}$ &(T-SEC)\\
\end{longtable}

% \begin{longtable}[c]{cr}
%   $\infer{\Gamma\vdash x:T}{x:T\in\Gamma &\Gamma\vdash T::*}$ &(T-VAR)\\
%   $\infer{\Gamma\vdash \lambda x:S.t\ :\Pi x:S.T}{\Gamma\vdash S::* &\Gamma,x:S\vdash t:T}$ &(T-ABS)\\
%   $\infer{\Gamma \vdash t_1\ t_2 : [x\mapsto t_2]T}{\Gamma\vdash t_1:\Pi x:S.T &\Gamma\vdash t_2:S}$ &(T-APP)\\
%   $\infer{\Gamma\vdash (t_1,t_2:\Sigma x:S.T)\ :\Sigma x:S.T}{\Gamma \vdash t_1:S &\Gamma\vdash t_2:[x\mapsto t_1]T}$ &(T-PAIR)\\
%   $\infer{\Gamma\vdash t.1:S}{\Gamma\vdash t:\Sigma x:S.T}$ &(T-PROJ1)\\
%   $\infer{\Gamma\vdash t.2:[x\mapsto t.1]T}{\Gamma\vdash t:\Sigma x:S.T}$ &(T-PROJ2)\\
%   $\infer{\Gamma\vdash t:T'}{\Gamma\vdash t:T &\Gamma\vdash T\equiv T'::*}$ &(T-CONV)\\
% \end{longtable}