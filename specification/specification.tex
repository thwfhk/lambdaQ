\documentclass[12pt]{article}

\usepackage[a4paper]{geometry}
\geometry{left=2.0cm,right=2.0cm,top=2.5cm,bottom=2.5cm}

\usepackage{comment}
\usepackage{booktabs}
\usepackage{graphicx}
\usepackage{diagbox}
\usepackage{amsmath,amsfonts,graphicx,amssymb,bm,amsthm}
\usepackage{algorithm,algorithmicx}
\usepackage[noend]{algpseudocode}
\usepackage{fancyhdr}
\usepackage{hyperref}
\usepackage{fontspec}
\usepackage{longtable}
\usepackage{xcolor}
\usepackage{newclude}

%Includes "References" in the table of contents
\usepackage[nottoc]{tocbibind}

%Import the natbib package and sets a bibliography style
\usepackage{natbib}

\newfontfamily{\lstfont}[Scale=.85]{Fira Sans}

\hypersetup{hidelinks}

\bibliographystyle{abbrvnat}
\usepackage{authblk}

% write inference rules
\usepackage{proof}
\usepackage{semantic}
\usepackage{mathpartir}

\newtheorem{Def}{Definition}
\newtheorem{Thm}{Theorem}
\newtheorem{Prop}{Proposition}
\newtheorem{Proof}{Proof}

% \usepackage{listings}
% \definecolor{light-gray}{gray}{0.85}
% \lstdefinelanguage{lambda}{%
%   morekeywords={%
%     if,then,else,fix,fun % keywords go here
%   },%
%   morekeywords={[2]Nat,Bool,Vector},   % types go here
%   otherkeywords={}, % operators go here
%   literate={% replace strings with symbols
%     {->}{{$\to$}}{2}
%     {lambda}{{$\lambda$}}{1}
%     {Pi}{{$\Pi$}}{1}
%     {Sigma}{{$\Sigma$}}{1}
%     {n1}{{$n_1$}}{1}
%     {n2}{{$n_2$}}{1}
%     {v1}{{$v_1$}}{1}
%     {v2}{{$v_2$}}{1}
%   },
%   basicstyle={\lstfont}, 
%   keywordstyle={\bfseries},
%   keywordstyle={[2]\itshape}, % style for types
%   keepspaces,
%   mathescape, 
%   columns=fixed,
%   backgroundcolor = \color{light-gray}
% }[keywords,comments,strings]


\title{The Definition of $\lambda_Q$}
\author{}
% \affil{Department of EECS\\ Peking University}
\date{\today}

\begin{document}

\maketitle
\tableofcontents

\renewcommand{\t}{\mathtt{t}}

\newcommand{\true}{\mathtt{true}}
\newcommand{\false}{\mathtt{false}}
\newcommand{\tif}{\mathtt{if}}
\newcommand{\tthen}{\mathtt{then}}
\newcommand{\telse}{\mathtt{else}}
\newcommand{\unit}{\mathtt{unit}}
\newcommand{\Unit}{\mathtt{Unit}}
\newcommand{\One}{\mathtt{1}}
\newcommand{\Bool}{\mathtt{Bool}}
\newcommand{\Bit}{\mathtt{Bit}}
\newcommand{\Qubit}{\mathtt{Qubit}}
\newcommand{\trun}{\mathtt{run}}
\newcommand{\toutput}{\mathtt{output}}
\newcommand{\from}{\leftarrow}
\newcommand{\From}{\Leftarrow}
\newcommand{\To}{\Rightarrow}
\newcommand{\gate}{\mathtt{gate}}
\newcommand{\lift}{\mathtt{lift}}
\newcommand{\capp}{\mathtt{capp}}
\newcommand{\calG}{\mathcal{G}}
\newcommand{\new}{\mathtt{new}}
\newcommand{\init}{\mathtt{init}}
\newcommand{\meas}{\mathtt{meas}}
\newcommand{\discard}{\mathtt{discard}}
\newcommand{\Ha}{\mathtt{H}}
\newcommand{\X}{\mathtt{X}}
\newcommand{\Z}{\mathtt{Z}}
\newcommand{\CNOT}{\mathtt{CNOT}}


\newcommand{\ccc}{\color[rgb]{0.5,0.11,0.1}}

\section{Syntax}

\begin{longtable}[c]{lclr}
  % \caption{syntax}
  \label{tab:table1}\\
  \toprule
  $t$ &$::=$ &  &\textbf{terms}: \\
      & &$x$ &variable\\
      & &$\unit$ &constant unit\\
      & &$\true$ &constant true\\
      & &$\false$ &constant false\\
      & &$\lambda\ x:T.t$ &function abstraction\\
      & &$t\ t$ &function application\\
      & &$(t, t)$ &pair\\
      & &$t.1$ &first projection\\
      & &$t.2$ &second projection\\
      & &$\mathit{if}\ t\ \mathit{then}\ t\ \mathit{else}\ t$ &conditional\\
      & &$\mathit{run}\ C$ &static lifting\\
      & &$\zeta \ p:W . C$ &circuit abstraction\\
      % & &0 &constant zero\\
      % & &succ $t$ &successor\\
      % & &pred $t$ &predecessor\\
      % & &iszero $t$ & zero test\\
  \\
  
  $v$ &$::=$ &  &\textbf{values}: \\
      & &$\lambda\ x:T.t$ &abstraction value\\
      & &$(v, v)$ &pair value\\
      & &$\unit$ &unit value\\
      & &$\true$ &true value\\
      & &$\false$ &false value\\
      & &$\zeta\ p:W . C$ &circuit value\\
  \\

  $T$ &$::=$ &  &\textbf{types}: \\
      & &$\Unit$ &unit type\\
      & &$\Bool$ &boolean type \\
      & &$T\times T$ &product type\\
      & &$T\to T$ &function type\\
      & &$T\leadsto T$ &circuit type\\
      % & &Nat &type of natural numbers\\
      % & &Vector $t$ &type family of vectors\\
  \\

  $\Gamma$ &$::=$ &  &\textbf{contexts}: \\
      & &$\varnothing$ &empty context\\
      & &$\Gamma,x:T$ &term variable binding\\
  \\


  $W$ &$::=$ &  &\textbf{wire types}: \\
      & &$One$ &wire unit type\\
      & &$Bit$ &bit type \\
      & &$Qubit$ &qubit type \\
      & &$W \otimes W$ &wire product type \\
  \\

  $p$ &$::=$ &  &\textbf{wire patterns}: \\
      & &$()$ &empty\\
      & &$w$ &wire variable \\
      & &$(p,p)$ &wire pair \\
  \\

  $C$ &$::=$ &  &\textbf{circuits}: \\
      & &$\mathit{output}\ p$ &output a pattern \\
      & &$p_2 \from \gate\ g\ p_1 ; C$ &gate application \\
      & &$p \from C ; C$ &circuit composition \\
      & &$x \From \lift\ p ; C$ &dynamic lifting \\
      & &$\unbox\ t\ p$ &circuit application \\
  \\

  $\Omega$ &$::=$ &  &\textbf{wire contexts}: \\
      & &$\varnothing$ &empty context\\
      & &$\Omega,w:W$ &wire variable binding\\
  \\
  % TODO: add the condition of well-formed contexts

  \bottomrule
  
\end{longtable}

\section{Type Checking Rules}
First, we need to define what is a well-formed wire context.
\begin{Def}[Well-formed Wire Contexts]
  A wire context $\Omega$ is well-formed, if there are no duplicate wire variables in it.
  For simplicity, we always assume the wire contexts are well-formed in the following contexts. And when we write $\Omega_1, \Omega_2$, we require $\Omega_1$ and $\Omega_2$ to be disjoint to preserve the well-formedness.
\end{Def}

Since there are three different kinds of terms: ($\lambda$-)terms, wire patterns and circuits, we have three different relations for each of them to check the correctness.
\begin{itemize}
  \item $\Omega \Rightarrow p : W$ is used to check the well-formedness of patterns.
  \item $\Gamma ; \Omega \vdash C : W$ is the typing relation for circuits;
  \item $\Gamma \vdash t:T$ is the typing relation for ($\lambda$-)terms;
\end{itemize}

\noindent \textbf{Well-formed patterns}: $\boxed{\Omega \Rightarrow p : W}$

\renewcommand\arraystretch{2.5}
\begin{longtable}[c]{cr}
  $ \infer{\varnothing \To ():One}{}$ & \\
  $ \infer{w:W \To w:Wz}{}$ & \\
  $ \infer{\Omega_1, \Omega_2 \Rightarrow (p_1,p_2):W_1\otimes W_2}{\Omega_1 \To p_1 : W_1  &\Omega_2 \To p_2 : W_2} $ & \\
\end{longtable}

\noindent \textbf{Typing rules for circuits}: $\boxed{\Gamma;\Omega\vdash C:W}$

{\color{red} TODO: add typing rules for circuits.}

% \begin{longtable}[c]{cr}
%   $\infer{\Gamma\vdash x:T}{x:T\in\Gamma &\Gamma\vdash T::*}$ &(T-VAR)\\
%   $\infer{\Gamma\vdash \lambda x:S.t\ :\Pi x:S.T}{\Gamma\vdash S::* &\Gamma,x:S\vdash t:T}$ &(T-ABS)\\
%   $\infer{\Gamma \vdash t_1\ t_2 : [x\mapsto t_2]T}{\Gamma\vdash t_1:\Pi x:S.T &\Gamma\vdash t_2:S}$ &(T-APP)\\
%   $\infer{\Gamma\vdash (t_1,t_2:\Sigma x:S.T)\ :\Sigma x:S.T}{\Gamma \vdash t_1:S &\Gamma\vdash t_2:[x\mapsto t_1]T}$ &(T-PAIR)\\
%   $\infer{\Gamma\vdash t.1:S}{\Gamma\vdash t:\Sigma x:S.T}$ &(T-PROJ1)\\
%   $\infer{\Gamma\vdash t.2:[x\mapsto t.1]T}{\Gamma\vdash t:\Sigma x:S.T}$ &(T-PROJ2)\\
%   $\infer{\Gamma\vdash t:T'}{\Gamma\vdash t:T &\Gamma\vdash T\equiv T'::*}$ &(T-CONV)\\
% \end{longtable}

\noindent \textbf{Typing rules for ($\lambda$-)terms}: $\boxed{\Gamma\vdash t:T}$

{\color{red} TODO: add typing rules for terms.}

% \begin{longtable}[c]{cr}
%   $\infer{\Gamma\vdash x:T}{x:T\in\Gamma &\Gamma\vdash T::*}$ &(T-VAR)\\
%   $\infer{\Gamma\vdash \lambda x:S.t\ :\Pi x:S.T}{\Gamma\vdash S::* &\Gamma,x:S\vdash t:T}$ &(T-ABS)\\
%   $\infer{\Gamma \vdash t_1\ t_2 : [x\mapsto t_2]T}{\Gamma\vdash t_1:\Pi x:S.T &\Gamma\vdash t_2:S}$ &(T-APP)\\
%   $\infer{\Gamma\vdash (t_1,t_2:\Sigma x:S.T)\ :\Sigma x:S.T}{\Gamma \vdash t_1:S &\Gamma\vdash t_2:[x\mapsto t_1]T}$ &(T-PAIR)\\
%   $\infer{\Gamma\vdash t.1:S}{\Gamma\vdash t:\Sigma x:S.T}$ &(T-PROJ1)\\
%   $\infer{\Gamma\vdash t.2:[x\mapsto t.1]T}{\Gamma\vdash t:\Sigma x:S.T}$ &(T-PROJ2)\\
%   $\infer{\Gamma\vdash t:T'}{\Gamma\vdash t:T &\Gamma\vdash T\equiv T'::*}$ &(T-CONV)\\
% \end{longtable}

\section{Operational Semantics}
{\color{red} TODO: add semantics.}

\end{document}